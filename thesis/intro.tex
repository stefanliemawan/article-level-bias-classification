\chapter{Introduction}
\label{cha:intro}

% Allsides \cite{allsides-2022-bias-definition} defines media bias as "The tendency of news media to report in a way that reinforces a viewpoint, worldview, preference, political ideology, corporate or financial interests, moral framework, or policy inclination, instead of reporting in an objective way (simply describing the facts)". This phenomenon has existed and researched since the 1950s (cite), highlighting its enduring presence and impact on public perception. Media bias can manifest in various forms, including the selection of stories, framing of issues, and choice of language. This project would mainly focus on text-level context bias: phrasing bias, spin bias, and statement bias, as described in \cite{spinde-2024-taxonomy}: statement Bias refers to “members of the media interjecting their own opinions into the text”, phrasing Bias is characterized by inflammatory words, i.e., non-neutral language, spin Bias describes a form of bias introduced either by leaving out necessary information.


%%% Local Variables: 
%%% mode: latex
%%% TeX-master: "thesis"
%%% End: 