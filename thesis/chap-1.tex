\chapter{Introduction}
\label{cha:1}

\section{Overview}

This thesis project is part of an Advanced Master of Artificial Intelligence programme—Speech and Language Technology, in collaboration with the Media Bias Group \cite{media-bias-group}, who provided the topic and additional guidance along the project.

\section{Motivation}

Media misinformation and manipulation are rampant in today's landscape. Despite being a major societal issue, there have been hardly enough resources and work dedicated to the realm of media bias and its broader context.

Considerable work has been done on fake news and its detection, media bias operates on a higher level and cannot be described the same simply fake news. Fake news involves intentionally spreading false information, whereas media bias refers to the distortion or manipulation of information by media outlets, which may or may not be intentional, to favor certain perspectives or agendas. Unlike fake news, media bias can influence public perception subtly through selective reporting, framing, or sourcing, making it a complex and challenging issue to address comprehensively.

To combat media bias effectively, it is crucial to develop robust methodologies and technologies for detecting and analyzing biased content across various media platforms. Additionally, raising awareness about media literacy and critical thinking skills can empower individuals to identify and navigate biased information effectively in today's media landscape. Ultimately, addressing media bias requires concerted efforts from researchers, policymakers, media organizations, and the public to promote transparency, accountability, and integrity in news reporting and consumption.

\section{Goals}

To aid in the work of media bias, the goals of this project are defined as follows:
\begin{enumerate}
    \item Build a reliable, annotated dataset comprised of article content and metadata.
    \item Propose a method to effectively represent articles in a space vector.
    \item Design a system that is able to effectively detect and classify bias on the article level of granularity.
    \item Validate the resulting system and dataset.
\end{enumerate}

\section{About the Media Bias Group}

The Media Bias Group \cite{media-bias-group} was established in mid-2020 by Timo Spinde during his pursuit of a Ph.D. in computer science, having been integrated into the topic since his undergraduate studies, with a vision to aid others perceive news in a more balanced and conscious manner. After a year of planning how a system could uncover bias on a vast scale encompassing millions of articles, he founded the group and forged connections with various partners, particularly those relevant to specific aspects of the project. In just one year, the project has garnered support from multiple other research groups, with around twenty students from seven countries joining to contribute to the system. Since 2021, the group has also begun offering its first Ph.D. positions.

The group is comprised of a collective of scholars across various fields such as Psychology, Linguistics, and Computer Science, with a shared goal to comprehend the factors influencing human perception of news content as biased or one-sided. Currently, the network includes six main researchers and coordinators, twenty-one professors and postdocs, as well as eight active students. Numerous publications related to media bias have been published through the network into major conferences such as EMNLP 2021 \cite{spinde-2021-babe}, along with dataset and benchmark creations.


%%% Local Variables: 
%%% mode: latex
%%% TeX-master: "thesis"
%%% End: 