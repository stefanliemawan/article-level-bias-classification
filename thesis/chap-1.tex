\chapter{Introduction}
\label{cha:1}

\section{Overview}

This thesis project is part of an Advanced Master of Artificial Intelligence programme—Speech and Language Technology, conducted in collaboration with the Media Bias Group \cite{media-bias-group}, who provided the topic and additional guidance along the project.

\section{Motivation and goals}

Media misinformation and manipulation are rampant in today's landscape. Despite being a major societal issue, there has been insufficient attention and resources dedicated to understanding and addressing media bias in its broader context. While considerable work has been done on fake news and its detection, media bias operates at a more nuanced level and cannot be simply equated with fake news. Fake news involves intentionally spreading false information, whereas media bias refers to the distortion or manipulation of information by media outlets, which may or may not be intentional, to favour certain perspectives or agendas. Unlike fake news, media bias can subtly influence public perception through selective reporting, framing, or sourcing, making it a complex and challenging issue to address comprehensively.

Existing works in media bias classification tend to have major drawbacks. Available datasets are low in quantity, containing only small number of samples, questionable annotations, and each having their own different types of bias, format, and metrics, making the task of training a model for media bias classification even more difficult. Classification is mostly done on the sentence-level with binary output, which is far too naive and may miss the broader context and impact conveyed by an entire article. One of the main problems with sentence-level classification is that different sentences within the same article can exhibit contradictory or varying biases. Media content is typically presented in the form of articles composed of multiple paragraphs, making article-level classification significantly more practical and valuable, in addition to offering a more nuanced and accurate perspective on media bias. Thus, detecting biases at the article-level allows for a more comprehensive understanding of the overall narrative and agenda presented by the media outlet. This approach considers the context and holistic viewpoint provided by the entire article, which is crucial for interpreting the nuanced ways in which biases can manifest. Moreover, article-level detection better aligns with how media consumers engage with and interpret information, as they often read articles in their entirety rather than focusing solely on isolated sentences.

To combat media bias effectively, it is crucial to develop robust methodologies and technologies for detecting and analysing biased content at the article-level. This approach will aid readers in recognising and identifying potential biases in the information they consume, thereby raising awareness of the associated dangers of media bias and promoting media literacy as well as critical thinking skills, which empower individuals to navigate information effectively in today's complex media landscape. Ultimately, addressing media bias requires coordinated efforts from researchers, policymakers, media organizations, and the public to uphold transparency, accountability, and integrity in news reporting and consumption.

Therefore, to contribute in the effort of addressing media bias, the objectives of this project are outlined as follows:
\begin{enumerate}
    \item Extend current media bias dataset to make it more suitable for article-level classification.
    \item Propose methods to effectively represent articles in a space vector.
    \item Build a system that is able to effectively detect and classify bias at the article-level.
    \item Evaluate results.
\end{enumerate}

\section{About the Media Bias Group}

The Media Bias Group \cite{media-bias-group} was established in mid-2020 by Timo Spinde during his pursuit of a Ph.D. in computer science, having been integrated into the topic since his undergraduate studies, with a vision to aid others perceive news in a more balanced and conscious manner. After a year of planning how a system could uncover bias on a vast scale encompassing millions of articles, he founded the group and forged connections with various partners, particularly those relevant to specific aspects of the project. In just one year, the project has garnered support from multiple other research groups, with around twenty students from seven countries joining to contribute to the system. Since 2021, the group has also begun offering its first Ph.D. positions.

The group is comprised of a collective of scholars across various fields such as Psychology, Linguistics, and Computer Science, with a shared goal to comprehend the factors influencing human perception of news content as biased or one-sided. Currently, the network includes six main researchers and coordinators, twenty-one professors and postdocs, as well as eight active students. Numerous publications related to media bias have been published through the network into major conferences such as EMNLP 2021 \cite{spinde-2021-babe}, along with dataset and benchmark creations.


%%% Local Variables: 
%%% mode: latex
%%% TeX-master: "thesis"
%%% End: 