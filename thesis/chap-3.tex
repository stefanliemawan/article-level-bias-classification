\chapter{Literature Review}
\label{cha:3}

\section{Dataset}

Available datasets for article-level media bias classification are quite limited, often in different formats or covering different types of bias:
\begin{itemize}
    \item The BASIL dataset \cite{fan-2019-basil} (300 articles) is the most commonly used bias dataset, containing 300 articles with both article-level and phrase-level annotations. This dataset focuses on informational bias in news articles as it appears more frequently than lexical bias. However, the obvious drawback of this dataset is the low amount of articles included as it is not nearly enough data to get a good working detection model.
    \item NLPCSS \cite{chen-2020-nlpcss} (6964 rows), annotated via Ad Fontes labels, contains article-level textual content with three bias labels (bias, neutral, or unknown). Focusing on political bias and unfairness, this can technically be a decent dataset for work for an article-level classification task.
    \item The BAT \cite{spinde-2023-bat} dataset (6345 rows, Ad Fontes labels) is the most suitable for article-level bias detection as it supplies both bias-score and reliability-score for the whole article. However, the dataset itself does not supply the textual content of the articles and therefore needs to be extended, something that I have been working on as well during my Master's Thesis this and last year.
\end{itemize}

Annotating bias is not a straightforward task. Traditionally, this is done by hiring experts and journalists to manually read and determine how biased the content is. As with \cite{spinde-2021-babe}, annotations are generally compiled and majority voted to achieve the final annotation given a particular text. It is important to use multiple annotators to minimise introducing another form of bias towards the dataset. Annotators' personal background moderately influenced their decisions and should be taken into consideration when building datasets, along with other factors such as topics, reading news habits, and honest mistakes \cite{spinde-2021-bias-words}. Clearly, this is not a cheap procedure and can be a huge bottleneck in creating a reliable media bias dataset.

Alternatively, websites such as Allsides and AdFontes have their own experts and annotations which can be crawled and made use of. Several papers and datasets \cite{spinde-2023-bat,chen-2020-nlpcss,kulkarni-2018-multi-view} have used this approach.  However, since these datasets depend on manual labeling from a third-party organisation, the selection of articles likely also introduces bias into the dataset \cite{spinde-2023-bat}.

\begin{comment}
News Unfold: (NewsUnravel shows that a user-centric approach to media bias data collection can re- turn reliable data while being scalable and evaluated as easy to use. NewsUnravel demonstrates that feedback mechanisms are a promising strategy to reduce data collection expenses and continuously update datasets to changes in context.) (Our approach augments dataset quality by significantly increasing inter-annotator agreement by 26.31\% and improving classifier performance by 2.49\%)
\end{comment}

\section{Classification}

Current State-of-the-Art in media bias detection typically employs a transformer-based approach, by fine-tuning or exploiting Large Language Models \cite{devlin-2019-bert, liu-2019-roberta}.

MAGPIE \cite{horych-2024-magpie} is the first large-scale, RoBERTa-based, multi-task learning (MTL) model dedicated to bias-related tasks, a promising approach for media bias detection and can be used to enhance the accuracy and efficiency of existing models. Using MAGPIE's context representation instead of BERT for media detection can potentially improve performance. However, the model is only trained for sentence-level classification and also outputs binary results.

Many past works on automatic detection of media bias typically use the BASIL dataset, operating on a sentence level and output binary result (either biased or not) \cite{maab-2023-lexical-bias-detection, maab-2023-target-aware, guo-2022-modeling, van-den-berg-2020-context,lee-2021-unifying,lei-2022-sentence,lei-2024-event-relation}, complemented by the lack of appropriate and adequate datasets with article-level annotations \cite{demidov-2023-political-bias-classification}. One of the main problems with detecting on sentence level is that multiple sentences within the same article could have opposite or different biases. As media content is often delivered in the form of articles containing a number of paragraphs, detection at the article level is far more useful and desirable.

There are currently only a few works on media bias and article level classification. Chen et al. \cite{chen-2020-detecting-media-bias-gaussian} utilised a Gaussian mixture model, incorporating probability distributions of frequency, positions, and sequential order of lexical and informational sentence-level bias to detect article-level bias. Kulkarni \cite{kulkarni-2018-multi-view} utilised attention mechanism to model a network structure of an article to classify political ideology. Outside of media bias, Su et al. \cite{su-2021-classifying} used chunking methods combined with encoder-based '[CLS] Pooling' to extract representations on the document-level.

\begin{comment}(I think there is a cite for different sentences opposite bias in an article, I remember reading).\end{comment}

Furthermore, The traditional approach of assigning only a single frame label to news articles remains overly simplistic, given that a standard news story often incorporates multiple viewpoints, arguments, or facets, each potentially carrying distinct connotations or framing \cite{vallejo-2023-connecting}. An integer score label would be a slightly better solution to represent bias from a text, although it is still hardly ideal.

Additionally, it would be good to consider the explainability of a model when classifying a bias, as we would need to not just get the result, but to understand why contents are considered bias. Existing media bias detection systems typically concentrate solely on predicting the likelihood of a certain text being biased, offering limited insights into the underlying reason behind the decision.

\section{Working with article-length text}

Article text length generally falls between medium to long sequences, as they are not as lengthy as legal documents or clinical studies that usually contain multiple pages of text. Most news articles stay between several hundred to several thousand words. There are several additional challenges when working with longer sequences as classification models that demonstrated significant results have been shown to perform poorly or even fail when they are tested on large documents \cite{wan-2019-long-length}.

The most straightforward approach of standard fine-tuning of an encoder-based language model is not necessarily effective for article-level processing as they are mostly only able to process a maximum of 512 tokens (BERT), prompting significant information loss. Several techniques have been attempted to address this issue such as the sliding window techniques, CLS techniques \cite{su-2021-classifying}, where you exploit the 'CLS' token from a BERT embedding, and other more sophisticated models \cite{kulkarni-2018-multi-view}.

Additionally, there are other LLMs similar to BERT that are designed to handle longer sequences: Longformer \cite{beltagy-2020-longformer}, BigBird \cite{zaheer-2021-bigbird}. However, these models have a different architecture than the original BERT, often do not consistently surpass the baseline models in classification performance, and only performed notably better than the baseline models on only two datasets \cite{park-2022-efficient}.


%%% Local Variables: 
%%% mode: latex
%%% TeX-master: "thesis"
%%% End: 
